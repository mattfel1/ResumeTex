%----------------------------------------------------------------------------------------
%	PREAMBLE
%----------------------------------------------------------------------------------------

\documentclass[letterpaper]{article}
\usepackage[pass]{geometry}
\usepackage{anyfontsize}
\usepackage{paralist}

\pagenumbering{gobble} % Turn off page numbering

\setlength{\voffset}{-.5in}
\setlength{\hoffset}{-.5in}
\setlength{\oddsidemargin}{0in}
\setlength{\headheight}{0in}
\setlength{\headsep}{0in}
\setlength{\topmargin}{0in}

\setlength{\marginparsep}{0in}
\setlength{\footskip}{0in}
\setlength{\marginparpush}{0in}
\setlength{\marginparwidth}{0in}

\setlength{\textheight}{10in}
\setlength{\textwidth}{7.5in}

\setlength{\paperwidth}{8.5in}
\setlength{\paperheight}{11in}

\let\oldcompactitem\compactitem \renewcommand{\compactitem}{ \oldcompactitem \setlength{\itemsep}{-11pt} \setlength{\parskip}{0pt} \setlength{\itemindent}{20pt} \setlength{\parsep}{0pt} }

\begin{document}


%----------------------------------------------------------------------------------------
%	NAME AND STUFF
%----------------------------------------------------------------------------------------

\noindent {
\begin{center}
\LARGE\textbf{Matthew Feldman}
\normalsize\\11595 Waterbend Ct., Wellington, FL 33414\\561-307-1591  -  mattfel@stanford.edu
\noindent\rule{7.5in}{1pt}
\end{center}
}

%----------------------------------------------------------------------------------------
%	EDUCATION
%----------------------------------------------------------------------------------------

\noindent{\Large\textbf {EDUCATION\\}}
\indent{\normalsize \textbf{Doctor of Philosophy}, Electrical Engineering \hfill\hfill May 2020\\ }
\indent{\normalsize \textbf{Master of Science}, Electrical Engineering\\}
\indent{\normalsize Stanford University, Stanford, CA\\}

\indent{\normalsize \textbf{Bachelor of Science}, Electrical Engineering \hfill\hfill May 2014\\ }
\indent{\normalsize University of Florida, Gainesville, FL}

%----------------------------------------------------------------------------------------
%	ACADEMIA
%----------------------------------------------------------------------------------------

\noindent{\Large\textbf {\\ACADEMIA}}


\indent{\normalsize \textbf{Teaching Assistant}, Linear Control Systems Course and Lab \hfill\hfill August 2014 - December 2014\\ }
\indent{\normalsize University of Florida, Gainesville, FL}
\begin{compactitem}
	\item Conducted weekly lab sessions for students to gain experience using Matlab for linear controls applications\\
	\item Taught students basic concepts, such as state space system modeling and lead and lag controller design\\
	\item Graded homework and exams
\end{compactitem}

\indent{\normalsize \textbf{Computer Vision Programmer}, Machine Intelligence Lab \hfill\hfill August 2014 - December 201\\ }
\indent{\normalsize University of Florida, Gainesville, FL}
\begin{compactitem}
	\item Designed and implemented SLAM algorithms through visual and odometer sensor fusion to assist a mobile robot navigate a course for the IEEE Autonomous Robot competition\\
	\item Produced an undergraduate thesis on computer vision, Kalman filtering, and perspective geometry
\end{compactitem}

\indent{\normalsize \textbf{Optics in the City of Light REU Researcher}, Biophotonics Group \hfill\hfill June 2013 - July 2013\\ }
\indent{\normalsize Institut d'Optique, Palaiseau, France}
\begin{compactitem}
	\item Constructed 3-dimension Full-Field Optical Coherence Tomography setup to support a cell-level biological study\\
	\item Characterized spherical aberration and image quality degradation as a function of conjugation position by programming LabVIEW control system and Matlab data-processing script
\end{compactitem}

\indent{\normalsize \textbf{NanoJapan REU Researcher}, Ajayan Lab \hfill\hfill June 2012 - July 2012\\ }
\indent{\normalsize Rice University, Houston, TX}
\begin{compactitem}
	\item Enhanced batteries and supercapacitors by creating new nanostructures and graphene coating using chemical vapor deposition\\
	\item Grew and transferred graphene samples for international collaboration projects on graphene devices
\end{compactitem}

\indent{\normalsize \textbf{EEREU Researcher}, Materials Research Institute \hfill\hfill June 2011 - July 2011\\ }
\indent{\normalsize Pennsylvania State University, State College, PA}
\begin{compactitem}
	\item Designed and fabricated tunable microchip coils, using CST Microwave Studio to assess model feasibility and a Vector Network Analyzers for hardware testing\\ 
	\item Scanned small-scale phantoms using an MRI machine and newly-designed 600MHz microchips to improve tools available to biologists and antenna designers, with results published in yearly journal
\end{compactitem}

\indent{\normalsize \textbf{Research Assistant}, Instrumentation and Imaging Laboratory for Biomechanics \hfill\hfill January 2011 - May 2012\\ }
\indent{\normalsize University of Florida, Gainesville, FL}
\begin{compactitem}
	\item Created and debugged LabVIEW programs that model the kinematics of multi-joint mechanical arms for National Instruments’ database\\
	\item Modeled a functioning Klann Linkage system with dimensions similar to those of a “StrandBeest”\\
	\item Constructed and developed software to control a pneumatic Instron tensile stress machine from basic components to be used in future engineering courses at the university
\end{compactitem}

%----------------------------------------------------------------------------------------
%	INDUSTRY
%----------------------------------------------------------------------------------------

\noindent{\Large\textbf {\\INDUSTRY}}

\indent{\normalsize \textbf{Student Technical Assistant}, MIT Lincoln Laboratory \hfill\hfill January 2015 - August 2015\\ }
\indent{\normalsize Lexington, MA}
\begin{compactitem}
	\item Designed simulations in C++, Matlab, and Java to thoroughly test the next generation flight collision avoidance system called ACAS-Xu, for manned and unmanned aircraft\\
	\item Used the Grid supercomputer to run millions of flight simulations in parallel, and analyzed the output data to make assessments of the operational suitability of various collision-avoidance logic\\
	\item Designed new portable algorithms for the surveillance and tracking modules on board unmanned aircraft
\end{compactitem}

\indent{\normalsize \textbf{Avionics Hardware Development and Integration Intern}, SpaceX \hfill\hfill August 2012 - August 2014\\ }
\indent{\normalsize Hawthorne, CA}
\begin{compactitem}
	\item Developed Altium extensions in C\# and Python with unsupervised learning algorithms for streamlining the avionics design process\\
	\item Worked on thermal imaging systems on Falcon 9 Reusable to improve reliability and reduce cost\\
	\item Designed harnesses and data acquisition circuit boards for flight on Falcon 9 Reusable and Dragon\\
	\item Compiled data on various electronic interfaces for all current and future satellite missions\\
	\item Developed and qualified proprietary avionics systems to improve safety and reliability of all future Falcon 9 and Falcon Heavy flights, using Matlab, C++, and Bash
\end{compactitem}

\end{document}




% Engineering and Science Tutor, instaEDU.com					                       May 2013 – Present
% Gainesville, FL
% •	Taught science, math, and engineering concepts to students ranging in age from middle school to college
% •	Designed and developed a proof-of-concept math training resource to visually teach students about solving equations
% Sponsored Engineer, Integrated Product and Process Design Program	                            August 2013 – May 2014
% 	Stryker Sustainability Solutions at University of Florida, Gainesville, FL
% •	Lead and worked with in a multidisciplinary team of engineers
% •	Designed, manufactured, and tested a C-based embedded system and fixture to rapidly test the integrity of the circuitry inside a particular ultrasonic scalpel surgery tool
% Director of Energy and Environment, The Dynamo Policy Research Group                   September 2010 – May 2012
% 	University of Florida, Gainesville, FL
% •	Published a policy recommendation on Smart Grid Systems in the “10 Ideas- Energy and Environment” publication and Roosevelt Institute’s peer-reviewed “Solutions for the South” online publication, where policy makers are known to extract ideas
% •	Discussed political topics regarding Energy and Environment via the Dynamo’s blog for the university community to read and consider
% •	Hosted an expert forum on Technological Innovations in Education at the University of Florida
% LEADERSHIP
% Founder, “Five for Tanzania” Charity Fundraiser for Rhotia Valley, Tanzania      	             September 2010 - Present
% 	University of Florida
% •	Raised donations and suppor for the Rhotia Valley children’s home and for tsunami victims in Minamisanriku, Japan from the publicity of setting multiple world records in the sport of “joggling,” or running and juggling at the same time
% Vice President, “Objects in Motion” (Juggling Club)                                        	              August 2010 – May 2011
% 	University of Florida
% •	Designed novel juggling props and developed mass production techniques 
% •	Designed choreography for live performances in Gainesville
% Space Florida Academy, NASA-oriented engineering program sponsored by Lockheed Martin	        March 2011
% Cape Canaveral, FL
% •	Designed, constructed, and launched a weather balloon payload during the week of Spring break with numerous other engineers from Florida in order to stream images of Earth from the stratosphere
% •	Worked and interacted with engineers and physicists from NASA, Lockheed Martin, and United Launch Alliance throughout multiple panel discussions
% ACHIEVEMENTS
% Undergraduate financed 100% of college tuition with merit-based scholarships	                  August 2010 - present
% Guinness World Record Holder, Fastest 400m, mile, and 5k while juggling 5 objects	         July 2011 – present
% Commissioned Student Ambassador to Miyazu, Japan for the city of Delray Beach, FL          April 2008 – June 2010
% AFFILIATIONS
% Member, IEEE Professional Engineering Society					                 October 2010 – Present
% 	Member, Student Small Satellite Design Club				             November 2010 – December 2011
% 	Benton Engineering Council Representative, Gator Amateur Radio Club                 January 2011 – December 2011
% 	Licensed Amateur Radio Technician							   January 2011 – Present
% PUBLICATIONS
% •	Feldman M, Lanagan M, Perini S. MRI microcoils for imaging individual cells. Annual Research Journal Electrical Engineering Research Experience for Undergrads. IX:169-179, 2011 August
% •	Legel L, Feldman M. Smart grid deployment plans for Florida’s utilities. 10 Ideas for Energy & Environment. 14-15, 2011 July
% •	Feldman M, Gullapalli H, Reddy LM, Vajtai R, Ajayan PM. Fluorine-etched nanostructures for energy storage applications.  RQI Symposium. Rice University, 2012 August 3.
