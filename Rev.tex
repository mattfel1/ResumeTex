%----------------------------------------------------------------------------------------
%	PREAMBLE
%----------------------------------------------------------------------------------------

\documentclass[letterpaper]{article}
\usepackage[pass]{geometry}
\usepackage{anyfontsize}
\usepackage{paralist}

\pagenumbering{gobble} % Turn off page numbering

\setlength{\voffset}{-.5in}
\setlength{\hoffset}{-.5in}
\setlength{\oddsidemargin}{0in}
\setlength{\headheight}{0in}
\setlength{\headsep}{0in}
\setlength{\topmargin}{0in}

\setlength{\marginparsep}{0in}
\setlength{\footskip}{0in}
\setlength{\marginparpush}{0in}
\setlength{\marginparwidth}{0in}

\setlength{\textheight}{10in}
\setlength{\textwidth}{7.5in}

\setlength{\paperwidth}{8.5in}
\setlength{\paperheight}{11in}

\let\oldcompactitem\compactitem \renewcommand{\compactitem}{ \oldcompactitem \setlength{\itemsep}{-11pt} \setlength{\parskip}{0pt} \setlength{\itemindent}{20pt} \setlength{\parsep}{0pt} }

\begin{document}


%----------------------------------------------------------------------------------------
%	NAME AND STUFF
%----------------------------------------------------------------------------------------

\noindent {
\begin{center}
\LARGE\textbf{Matthew Feldman}
\normalsize\\11595 Waterbend Ct., Wellington, FL 33414\\561-307-1591  -  mattfel@stanford.edu
\noindent\rule{7.5in}{1pt}
\end{center}
}

%----------------------------------------------------------------------------------------
%	EDUCATION
%----------------------------------------------------------------------------------------

\noindent{\Large\textbf {EDUCATION\\}}
\indent{\normalsize \textbf{Doctor of Philosophy}, Electrical Engineering \hfill\hfill May 2020\\ }
\indent{\normalsize \textbf{Master of Science}, Electrical Engineering\\}
\indent{\normalsize Stanford University, Stanford, CA\\}

\indent{\normalsize \textbf{Bachelor of Science}, Electrical Engineering \hfill\hfill December 2014\\ }
\indent{\normalsize University of Florida, Gainesville, FL}

%----------------------------------------------------------------------------------------
%	ACADEMIA
%----------------------------------------------------------------------------------------

\noindent{\Large\textbf {\\EXPERIENCE}}


\indent{\normalsize \textbf{Computer Vision Programmer}, Machine Intelligence Lab \hfill\hfill August 2014 - December 201\\ }
\indent{\normalsize University of Florida, Gainesville, FL}
\begin{compactitem}
	\item Designed and implemented SLAM algorithms through visual and odometer sensor fusion to assist a mobile robot navigate a course for the IEEE Autonomous Robot competition\\
	\item Produced an undergraduate thesis on computer vision, Kalman filtering, and perspective geometry
\end{compactitem}

\indent{\normalsize \textbf{Student Technical Assistant}, MIT Lincoln Laboratory \hfill\hfill January 2015 - August 2015\\ }
\indent{\normalsize Lexington, MA}
\begin{compactitem}
	\item Developed surveillance metrics and software in Matlab to rapidly automate the testing of tracking algorithms and parameters between new surveillance modules and legacy systems\\
	\item Wrote parallel Matlab for the grid supercomputer to simulate thousands of random airspace environments for testing the tracking system and tens of thousands of encounter geometries for analyzing the operational suitability of the collision avoidance logic\\
	\item Characterized the dynamics of General Atomics' aircraft to develop recommendations for collision avoidance logic.\\
	\item Uncovered bugs in existing algorithms and tweaked them accordingly for the surveillance and tracking modules on-board unmanned aircraft
\end{compactitem}

\indent{\normalsize \textbf{Avionics Hardware Development and Integration Intern}, SpaceX \hfill\hfill August 2012 - August 2014\\ }
\indent{\normalsize Hawthorne, CA}
\begin{compactitem}
	\item Developed Altium extensions in C\# and Python with unsupervised learning algorithms for streamlining the avionics design process\\
	\item Worked on thermal imaging systems on Falcon 9 Reusable to improve reliability and reduce cost\\
	\item Designed harnesses and data acquisition circuit boards for flight on Falcon 9 Reusable and Dragon\\
	\item Compiled data on various electronic interfaces for all current and future satellite missions\\
	\item Developed and qualified proprietary avionics systems to improve safety and reliability of all future Falcon 9 and Falcon Heavy flights, using Matlab, C++, and Bash
\end{compactitem}

\indent{\normalsize \textbf{Engineering and Science Tutor}, InstaEDU.com \hfill\hfill May 2013 - Present\\ }
\indent{\normalsize Gainesville, FL}
\begin{compactitem}
	\item Taught science, math, and engineering concepts to students ranging in age from middle school to college\\
	\item Designed and developed a proof-of-concept math training resource to visually teach students about solving equations
\end{compactitem}

\indent{\normalsize \textbf{Sponsored Engineer}, Integrated Product and Process Design Program \hfill\hfill August 2013 - May 2014\\ }
\indent{\normalsize Stryker Sustainability Solutions at University of Florida, Gainesville, FL}
\begin{compactitem}
	\item Lead and worked in a multidisciplinary team of engineers\\
	\item Designed, manufactured, and tested a C-based embedded system and fixture to rapidly test the integrity of the circuitry inside a particular ultrasonic scalpel surgery tool
\end{compactitem}

\indent{\normalsize \textbf{Director of Energy and Environment}, The Dynamo Policy Research Group \hfill\hfill September 2010 - May 2012\\ }
\indent{\normalsize Stryker Sustainability Solutions at University of Florida, Gainesville, FL}
\begin{compactitem}
	\item Published a policy recommendation on Smart Grid Systems in the “10 Ideas- Energy and Environment” publication and Roosevelt Institute’s peer-reviewed “Solutions for the South” online publication, where policy makers are known to extract ideas\\
	\item Discussed political topics regarding Energy and Environment via the Dynamo’s blog for the university community to read and consider\\
	\item Hosted an expert forum on Technological Innovations in Education at the University of Florida\\\\
\end{compactitem}

%----------------------------------------------------------------------------------------
%	LEADERSHIP
%----------------------------------------------------------------------------------------

\noindent{\Large\textbf {\\LEADERSHIP}}

\indent{\normalsize \textbf{Founder}, “Five for Tanzania” Charity Fundraiser for Rhotia Valley, Tanzania \hfill\hfill September 2011\\ }
\indent{\normalsize University of Florida, Gainesville, FL}
\begin{compactitem}
	\item Raised donations and support for the Rhotia Valley children’s home and for tsunami victims in Minamisanriku, Japan from the publicity of setting multiple world records in the sport of “joggling,” or running and juggling at the same time
\end{compactitem}

\indent{\normalsize \textbf{Vice President}, "Objects in Motion" (Juggling Club) \hfill\hfill August 2010 - May 2011\\ }
\indent{\normalsize University of Florida, Gainesville, FL}
\begin{compactitem}
	\item Designed novel juggling props and developed mass production techniques\\
	\item Designed choreography for live performances in Gainesville
\end{compactitem}


%----------------------------------------------------------------------------------------
%	AFFILIATIONS
%----------------------------------------------------------------------------------------

\noindent{\Large\textbf {\\AFFILIATIONS}}

\indent{\normalsize {\textbf {Engineer-in-Training}, National Council of Examiners for Engineering and Surveying \hfill\hfill August 2015 - present\\}
\indent{\normalsize {\textbf {Member}, IEEE Professional Engineering Society \hfill\hfill July 2011 - present\\}
\indent{\normalsize {\textbf {Licensed Amateur Radio Technician} \hfill\hfill January 2011 - Present}


%----------------------------------------------------------------------------------------
%	PUBLICATIONS
%----------------------------------------------------------------------------------------

\noindent{\Large\textbf {\\PUBLICATIONS}}

\begin{compactitem}
	\item \textbf {Feldman M}, Lanagan M, Perini S. MRI microcoils for imaging individual cells. Annual Research Journal Electrical Engineering Research Experience for Undergrads. IX:169-179, 2011 August\\
    \item Legel L, \textbf {Feldman M}. Smart grid deployment plans for Florida’s utilities. 10 Ideas for Energy \& Environment. 14-15, 2011 July\\
	\item \textbf {Feldman M}, Gullapalli H, Reddy LM, Vajtai R, Ajayan PM. Fluorine-etched nanostructures for energy storage applications.  RQI Symposium. Rice University, 2012 August 3.
\end{compactitem}


\end{document}




